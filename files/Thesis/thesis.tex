% !TEX TS-program = lualatex

\documentclass[oneside, a5paper,10pt]{article}
\usepackage{System/core}

\newcommand*\rfrac[2]{{}^{#1}\!/_{#2}}
\renewcommand{\proofname}{Доказательство}

\newcounter{num_authors}
\setcounter{num_authors}{0}

\newarray\authorName
\newarray\authorOrganisation
\newarray\authorMail

%%\isequal{что сравнить}{с чем сравнить}{если да}{если нет}
%%делает проверку, одинаковы ли макросы "что сравнить" и "с чем сравнить". Если одинаковы - выполняет "если ла", в противном случае - "если нет"
%Команда бесполезна.
\newcommand{\isequalmacros}[4]{
\def\a{#1}\def\b{#2}\ifx\a\b#3\else#4\fi
}

\def\ifEqStringBase#1#2%
{\def\csa{#1}\def\csb{#2}\ifx\csa\csb}

%\ifEqString{СТРОКА1}{СТОКА2}{УС_ДА}{УС_НЕТ}
%сравнивает строки СТРОКА1 и СТРОКА 2. Если равны, то выполняет УС_ДА, если нет -- УС_НЕТ.
\newcommand{\ifEqString}[4]{\ignorespaces
\ifEqStringBase{#1}{#2}#3\else#4\fi
}


%%\addAuthor{ФИО}{организация}{e-mail}
%%Добавляет автора
%%Нумерация массива - С ЕДИНИЦЫ!!
\newcommand{\addAuthor}[3]{
\addtocounter{num_authors}{1}
\authorName(\arabic{num_authors})={#1}
\authorOrganisation(\arabic{num_authors})={#2}
\authorMail(\arabic{num_authors})={#3}
}

%\roflll{\x}
\newcommand{\roflll}[1]{
\ignorespaces
\checkauthorMail(#1)\def\mystring{\cachedata}\StrLen{\mystring}[\mystringlenn]\ifnum\mystringlenn>0\checkauthorName(#1)\cachedata$^{*\text{#1}}$\else\checkauthorName(#1)\cachedata$^{\text{#1}}$\fi}

\newcommand{\makeAuthorList}{
\ignorespaces
\begin{center}
\ifthenelse{\arabic{num_authors} = 0}{}{
\foreach\x in{1,...,\arabic{num_authors}}{
\roflll{\x}\ifnum\x<\arabic{num_authors}{,}\fi}
}
\end{center}
\vspace{-7mm}
\begin{center}
\foreach\x in{1,...,\arabic{num_authors}}{
\textit{$^{\text{\x}}$\checkauthorOrganisation(\x)\cachedata}\ifnum\x<\arabic{num_authors}{,}\fi\\
}
\end{center}
\vspace{-5mm}
\begin{center}
\foreach\x in{1,...,\arabic{num_authors}}{
\checkauthorMail(\x)\def\mystring{\cachedata}\StrLen{\mystring}[\mystringlenn]\ifnum\mystringlenn>0$^{\text{*}}$e-mail:\,\textit{\cachedata}\fi
}
\end{center}
}

\pagestyle{fancy}
\fancyhf{}
\chead{
Международная Зимняя научная сессия СНО НИЯУ МИФИ -- 2024
}
\cfoot{\thepage}

%%\makeInf{название}{тект аннотации}{Ключевые слова}{Тезисы}
\newcommand{\makeInf}[4]{
\renewcommand{\abstractname}{\normalsize Аннотация}
%%%%%%%%%%%%%%%%%%%%%%%%%%%%%%%%%%%%%%
\begin{center}
\bfseries\large{#1}
\end{center}
%%%%%%%%%%%%%%%%%%%%%%%%%%%%%%%%%%%%%
\vspace{-5mm}
\makeAuthorList
%%%%%%%%%%%%%%%%%%%%%%%%%%%%%%%%%%%%%%
\begin{abstract}
\begin{minipage}{84mm}
\par
#2
%%%%%%%%%%%%%%%%%%%%%%%%%%%%%%%%%%%%%%%%
\vspace{8mm}
\par
#3
\end{minipage}
\end{abstract}

%%%%%%%%%%%%%%%%%%%%%%%%%%%%%%%%%%%%%%%%%%%%
\vspace{6mm}
\par
#4
}

\newcommand{\Picture}[3]{%
\begin{figure}[H]
\centering
  \includegraphics[keepaspectratio, width=#3\textwidth]{Pictures/#1}
\caption{#2}\label{#1}
\end{figure}
}

\makeatletter
\renewenvironment{thebibliography}[1]
{ %\refname
	\smallskip
	\nopagebreak
	\centerline{\\ \textbf{\refname}}
	\nopagebreak
	\smallskip
	\@afterheading
	
	%\@mkboth{\MakeUppercase\refname}{\MakeUppercase\refname}%
	\list{
		\@biblabel{\@arabic\c@enumiv}
	}%
	{
		\settowidth\labelwidth{\@biblabel{#1}}%
		\leftmargin\labelwidth
		\advance\leftmargin\labelsep
		\@openbib@code
		\usecounter{enumiv}%
		\let\p@enumiv\@empty
		\renewcommand\theenumiv{\@arabic\c@enumiv}}%
	\sloppy
	\clubpenalty4000
	\@clubpenalty \clubpenalty
	\widowpenalty4000%
	\sfcode`\.\@m}
{\def\@noitemerr
	{\@latex@warning{Empty `thebibliography' environment}}%
	\endlist}
\makeatother

\makeatletter
\renewcommand\@biblabel[1]{#1.}
\makeatother

\newcommand{\p}{\partial}

\begin{document}

\addAuthor{\textbf{\small Донецков Андрей Дмитриевич}}{НИЯУ МИФИ}{andrey.donetskov@gmail.com}
\addAuthor{Бакакин Валерий Дмитриевич}{}{}
\addAuthor{Жулев Егор Михайлович}{}{}

\makeInf
{Влияние формулировки задачи на решение методом PINN}
{Целью работы является анализ влияния различных формулировок задачи Коши для уравнения гармонического осциллятора с вынуждающей силой на эффективность и точность решений. Проведённые эксперименты показали влияние постановки задачи на сходимость нейронной сети и вычислительные затраты.}
{Ключевые слова: гармонический осциллятор, PINN.}
{\section*{Постановка задачи}
Рассматриваются три постановки задачи Коши:

\textbf{1. ОДУ второго порядка:}
\begin{equation}
\label{DonetskovAD-ODE2}
\frac{d^2x}{dt^2} + \omega_0^2 x = -A\cos(\omega t), \quad x(0) = x_0, \quad \frac{dx}{dt}(0) = v_0.
\end{equation}

\textbf{2. Система ОДУ первого порядка:}
\begin{equation}
\label{DonetskovAD-ODE1}
\frac{dx}{dt} = y, \quad \frac{dy}{dt} = -\omega_0^2 x - A\cos(\omega t), \quad x(0) = x_0, \quad y(0) = v_0.
\end{equation}

\textbf{3. Альтернативная система ОДУ:}
\begin{equation}
\label{DonetskovAD-AltODE}
\frac{dx}{dt} = \omega y - \frac{A}{\omega} \sin(\omega t), \quad \frac{dy}{dt} = -\omega x, \quad x(0) = x_0, \quad y(0) = \frac{v_0}{\omega}.
\end{equation}

\section*{Методология}
Физически-информированные нейронные сети (PINN) используются для аппроксимации решений \cite{bibl:raissi2019}. Функция потерь минимизирует отклонения от исходных уравнений и начальных условий \cite{bibl:lagaris1998}.

\section*{Результаты}
\Picture{stitched_image}{Функция потерь для альтернативной системы (слева), системы ОДУ первого порядка (в центре) и второго порядка (справа).}{0.6}

Анализ графиков функции потерь показывает:
\begin{itemize}
    \item \textbf{ОДУ второго порядка} показывает наибольшую скорость сходимости и высокую точность, но часто выходит из локальных минимумов;
    \item \textbf{Система ОДУ первого порядка} демонстрирует медленную скорость сходимости и меньшую точность, но стабильный процесс обучения;
    \item \textbf{Альтернативная система ОДУ} обеспечивает точность, сопоставимую с первым случаем, при меньших колебаниях функции потерь.
\end{itemize}

Таким образом, выбор формулировки задачи влияет на характер сходимости и вычислительные затраты.

\begin{thebibliography}{9}
    \bibitem{bibl:lagaris1998} \textit{Lagaris I.E., Likas A., Fotiadis D.I.} Artificial neural networks for solving ordinary and partial differential equations // IEEE Transactions on Neural Networks. \textbf{1998}. \textit{9}(5): 987-1000.
    \bibitem{bibl:raissi2019} \textit{Raissi M., Perdikaris P., Karniadakis G.E.} Physics-informed neural networks: A deep learning framework for solving forward and inverse problems involving nonlinear partial differential equations // Journal of Computational Physics. \textbf{2019}. \textit{378}: 686-707.
\end{thebibliography}
}
\end{document}
